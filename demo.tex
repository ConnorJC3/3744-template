\documentclass{3744lab}
\author{Last Name, First Name}
\labnumber{\textcolor{wordpurple}{\#}}
\title{\textcolor{wordpurple}{Lab Title}}
\classnumber{<5-digit class number>}
\usepackage[T1]{fontenc}
\begin{document}
  \begin{notmet}
    \noindent <insert any requirements not met, if applicable (if not applicable, write “N/A”)>
  \end{notmet}
  \begin{problems}
    \noindent <insert a brief summary of \textit{all} problems encountered>
  \end{problems}
  \begin{future}
    \noindent <insert a brief paragraph describing how the topics covered in this lab could potentially be used for other \linebreak applications>
  \end{future}
  \begin{prelab}
    \noindent <insert copy of pre-lab exercises from lab document, as well as an answer directly following each of the \linebreak questions (if not applicable, write “N/A”)>
  \end{prelab}
  \begin{codecharts}
    \subsection{(1, 2, etc.)}
    \noindent <insert easily readable pseudocode/flowcharts, when applicable, clearly distinguishing between each part of the lab (write “N/A” if there are none)>
  \end{codecharts}
  \begin{programcode}
    \subsection{(1, 2, etc.)}
    \noindent <insert copy of all required main program code, clearly distinguishing between each part of the lab (write “N/A” if there are none)> 
  \end{programcode}
  \begin{appendix}
    \noindent <insert copy of \textit{all} supporting ASM or C program code, e.g., header files referenced within your programs, as \linebreak well as any other relevant information, e.g., screenshots (with meaningful captions), when applicable (if not applicable, write “N/A”)>  
  \end{appendix}
\end{document}
